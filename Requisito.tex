\documentclass[article, 12pt, oneside, a4paper, brazil]{abntex2}

\usepackage{lmodern}
\usepackage[T1]{fontenc}
\usepackage[utf8]{inputenc}
\usepackage{nomencl}
\usepackage{color}
\usepackage{graphicx}
\usepackage{indentfirst}
\usepackage{microtype}
\usepackage{morefloats}

\usepackage[brazil]{babel}
\addto\captionsbrazil{
    \renewcommand{\bibname}{Refer\^encias}
    \renewcommand{\indexname}{\'Indice}
    \renewcommand{\listfigurename}{Lista de ilustra\c{c}\~{o}es}
    \renewcommand{\listtablename}{Lista de tabelas}
    \renewcommand{\pageautorefname}{p\'agina}
    \renewcommand{\sectionautorefname}{se{\c c}\~ao}
    \renewcommand{\subsectionautorefname}{subse{\c c}\~ao}
    \renewcommand{\paragraphautorefname}{par\'agrafo}
    \renewcommand{\subsubsectionautorefname}{subse{\c c}\~ao}
    \renewcommand{\paragraphautorefname}{subse{\c c}\~ao}
}

\usepackage{color}
\definecolor{thered}{rgb}{0.65,0.04,0.07}
\definecolor{thegreen}{rgb}{0.06,0.44,0.08}
\definecolor{thegrey}{gray}{0.5}
\definecolor{theshade}{rgb}{1,1,0.97}
\definecolor{theframe}{gray}{0.6}
\definecolor{blue}{RGB}{41,5,195}

\setlrmarginsandblock{3cm}{3cm}{*}
\setulmarginsandblock{3cm}{3cm}{*}
\checkandfixthelayout

\setlength{\parindent}{1.3cm}

\setlength{\parskip}{\onelineskip} 

\SingleSpacing


\begin{document}
 \section*{\textbf{Documento de Requisitos:} Sistema de Pedidos Eletrônico para Restaurantes com Controle de Fluxos de Produção e Programa de Fidelidade}
 \section{Introdução}
 
 \subsection{Propósito}
 O propósito deste documento de especificação de requisitos é definir todos os requisitos do Sistema de Pedidos Eletrônico para Restaurantes com Controle de Fluxos de Produção e Programa de Fidelidade, sistema que tem como objetivo principal gerenciar pedidos, controlar o fluxo de produção e informar os itens sendo produzidos aos clientes, bem como oferecer bonificações aos clientes baseado no histórico de compras.
 
 \subsection{Escopo}
 O sistema recebe os pedidos através de um atendente e os envia para a unidade produção, onde entram para o fluxo de produção de três estágios, sendo eles aguardando, em produção e pronto, sendo que tais informações são compartilhadas com os clientes.
 
 \subsection{Organização da Especificação de Requisitos de Software}
 
 Este documento está dividido em três seções. Na primeira seção, é apresentada uma breve introdução sobre o conteúdo deste documento. Na segunda seção, uma descrição geral do sistema é apresentada e na última seção são descritos, de forma detalhada, todos os requisitos funcionais e não-funcionais do sistema.
 
 \section{Descrição Geral do Sistema}
 O objetivo do sistema é receber os pedidos dos clientes através de um dispositivo móvel portado pelo atendente, o pedido é então analisado e enviado a unidade de produção, caso haja a falta de algum ingrediente para o preparo, o sistema deverá informar que o mesmo não pode ser pode ser produzido. 
 
 Após a aceitação do pedido, este é recebido na unidade de produção através do controle de fluxo de produção e entra para o estágio aguardando, assim que sua produção for iniciada, sua situação deve ser alterada para em produção e após finalizado para pronto. Tais informações também estarão disponíveis aos clientes por meio de um dispositivo de exibição. Uma cópia do pedido é impressa e entregue ao cliente e será utilizada para encerrar o processo.
 
 O pedido é finalizado quando o cliente efetua o seu pagamento, a identificação é realizada pelo número do pedido contido na cópia entregue a ele ou pelo número da mesa que estava ocupando, o cliente é identificado pelo seu número do CPF (Cadastro de Pessoas Físicas) e o pedido é salvo no histórico de compras. O histórico de compras visa oferecer bonificações baseadas nas regras de negócio.
 
 \subsection{Funções do Produto}
 O sistema apresenta como principal objetivo gerenciar o ciclo de produção em um restaurante, desde a realização até o pagamento do pedido, realizando as seguintes funções:
 
 \begin{itemize}
  \item Inclusão, alteração, exclusão e consulta de ingredientes;
  \item Inclusão, alteração, exclusão e consulta de produtos;
  \item Inclusão, alteração e consulta de pedidos;
  \item Inclusão e consulta de clientes;
  \item Emissão do histórico de compras por cliente;
 \end{itemize}

 \subsection{Características do Usuário}
 O sistema é destinado a três grupos de usuários, sendo eles: atendentes, cozinheiros e caixa. Sendo necessário ter uma noção básica sobre computadores.
 
 \subsection{Suposições e Dependências}
 A configuração mínima requerida para a execução do sistema é composta por dispositivos móveis portadores de android, dois microcomputadores, sendo um com tela sensível ao toque e o outro hospedando o sistema, por fim, uma televisão para exibir o estado dos pedidos.
 
 \section{Requisitos Específicos} 
 \subsection{Requisitos Funcionais}
 
 \subsection*{\emph{Cadastro de Ingredientes}}
 \begin{description}
  \item[RF1.] O sistema deve permitir a inclusão, alteração e remoção de ingredientes no sistema. Os dados de ingredientes consistem de: nome, preço, fornecedor, contato do fornecedor e quantidade em estoque.
  \item[RF2.] O sistema deve permitir o cadastro de apenas um ingrediente por nome.
  \item[RF3.] O sistema deve permitir apenas ao administrador incluir, alterar ou remover ingredientes.
 \end{description}

 \subsection*{\emph{Cadastro de Produtos}}
 \begin{description}
  \item [RF4.] O sistema deve permitir a inclusão, alteração e remoção de produtos no sistema. Os dados de produtos consistem de: número de identificação único, nome, preço, ingredientes e categoria.
  \item [RF5.] O sistema deve permitir a alteração dos dados do produto, exceto o número de identificação único.
  \item [RF6.] O sistema deve emitir mensagens de erro caso um produto seja adicionado aos pedidos e algum ingrediente esteja indisponível no estoque.
 \end{description}
 
 \subsection*{\emph{Cadastro de Pedidos}}
 \begin{description}
  \item [RF7.] O sistema deve permitir a inclusão e alteração dos pedidos. Os dados de pedido consistem de: número do pedido, produtos, valor total, identificação do cliente, data do pedido e estado do pedido.
  \item [RF8.] O sistema deve permitir a alteração dos dados do pedido, exceto o número do pedido, a identificação do cliente e a data do pedido.
  \item [RF9.] O sistema deve permitir a alteração do campo estado do pedido para: realizado, aprovado, aguardando, em produção, pronto e finalizado.
 \end{description}
 
 \subsection*{\emph{Cadastro de Clientes}}
 \begin{description}
  \item [RF10.] O sistema deve permitir a inclusão de clientes. Os dados de clientes consistem de: CPF, nome do cliente e pontos acumulados.
  \item [RF11.] O sistema deve permitir o cadastro de apenas um cliente por CPF.
 \end{description}
 
 \subsection*{\emph{Informações do Programa de Fidelidade}}
 \begin{description}
  \item [RF12.] O sistema deve permitir que um pedido esteja vinculado a apenas um cliente.
  \item [RF13.] O sistema deve permitir que apenas pedidos com o estado de finalizados sejam vinculados aos pontos acumulados do cliente.
 \end{description}
 
 \subsection*{\emph{Relatórios}}
 \begin{description}
  \item [RF14.] O sistema deve gerar relatórios de todos os pedidos realizados por cliente, data ou produto.
  \item [RF15.] O sistema deve gerar relatórios da quantidade de ingredientes em estoque.
 \end{description}
 
 \subsection{Requisitos Não-Funcionais}
 \begin{description}
  \item [RN1.] O sistema é composto por três (3) subsistemas, sendo eles: sistema para atendimento, sistema para fluxo de produção e sistema para caixa.
  \item [RN2.] O sistema deve ser capaz de realizar cópias de segurança de todos os dados do sistema.
  \item [RN3.] O sistema deve ser facilmente portável para os ambientes Linux e Windows.
 \end{description}


 
\end{document}
